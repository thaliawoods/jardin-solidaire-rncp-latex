\chapter{Présentation du projet (FR)}

\addcontentsline{toc}{section}{Présentation du projet (FR)}

Après plusieurs années en restauration, en vente et dans le social, et attirée depuis longtemps par le \textbf{développement web}, j’ai choisi en 2023 de me réorienter et de me former à ce métier.

J’ai intégré \textbf{Ada Tech School}, une formation en deux temps : 9 mois à l’école, puis 12 mois en alternance, menant au titre RNCP niveau 6 \textbf{Concepteur·rice Développeur·se d’Applications}. J’ai réalisé mon alternance chez \textbf{Julaya}, une fintech B2B, dans un environnement de production avec une stack moderne (Next.js, TypeScript, PostgreSQL) et une organisation agile.

Cette expérience m’a appris à travailler sur un produit réel : avancer par itérations, intégrer les retours, traiter les cas limites, et livrer du code \textbf{maintenable}.

C’est dans ce cadre que j’ai développé \textbf{Jardin Solidaire}.

À l’origine, je suis partie d’un constat simple : les espaces verts améliorent le bien-être et le lien social, mais beaucoup de jardins privés restent sous-utilisés. En parallèle, de nombreuses personnes aimeraient jardiner, apprendre, respirer, se reconnecter au vivant, sans forcément avoir accès à un jardin, surtout en ville. J’ai voulu construire une solution qui relie ces deux réalités, avec une approche \textbf{concrète} et \textbf{simple}.

\textbf{Jardin Solidaire} est une plateforme web d’entraide qui met en relation des \textbf{propriétaires de jardins} et des \textbf{jardinier.es volontaires}. L’idée n’est pas de créer un service marchand : il n’y a pas de paiement. Je mise sur le temps partagé, la coopération et la transmission, pour remettre en vie des jardins qui dorment et faciliter des rencontres de proximité.

Le projet a été initié à trois avec des camarades de ma promotion. À mi-parcours, nous avons choisi de nous séparer pour mener chacune une version aboutie, tout en gardant la même intention de départ.

Aujourd’hui, Jardin Solidaire permet de créer un profil propriétaire ou jardinier.e, de renseigner des informations utiles (photos, localisation, besoins ou compétences), de consulter un listing de jardins avec filtres et favoris, d’accéder à des fiches détaillées, et d’organiser une intervention grâce à des créneaux planifiés via un calendrier et une messagerie liée à l’échange.

Jardin Solidaire se situe au croisement de la \textbf{tech}, du \textbf{social} et de l’\textbf{écologie} : une solution simple pour créer du lien, partager des gestes, et prendre soin des espaces verts de quartier.
\newpage